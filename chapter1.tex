\documentclass{jsarticle}
\usepackage{amsmath}
\usepackage{bm}
\usepackage{cases}
\begin{document}

Chapter1の気温の問題で厳密解を求めてみよう。
\begin{equation}
	E(\omega_0,\omega_1,\omega_2,\omega_3,\omega_4)
	=\sum_{n=1}^{12} \left( \left( \sum_{m=0}^4 \omega_m n^m \right) - t_n \right) ^2 
\end{equation}

この関数の極値を求めたい。

極値をとる条件は、

\begin{equation*}
	\frac{\partial E}{\partial \omega_m}
	(\omega_0,\omega_1,\omega_2,\omega_3,\omega4)=0\:(m=0, \dots,4)
\end{equation*}

であり、この連立方程式を解けばよい。

$\displaystyle V_n=y_n-t_n=(\sum_{m=0}^{4} \omega_m n^m) - t_n$ として、

\begin{align*}
	& \frac{\partial E}{\partial \omega_m} 
	= \frac{\partial}{\partial \omega_m} (\sum_{n=1}^{12} V_n^2) \\
	&= \sum_{n=1}^{12} 
	\left( \frac{\partial} {\partial V_n} V_n^2 \right) 
	\frac{\partial V_n}{\partial y_n} \frac{\partial y_n}{\partial \omega_m} \\
	&= \sum_{n=1}^{12} (2V_n) n^m \\
	&= 2\sum_{n=1}^{12} \omega_0 n^m 
	+ \omega_1 n^{m+1} 
	+ \omega_2 n^{m+2} 
	+ \omega_3 n^{m+3} 
	+ \omega_4 n^{m+4} 
	- t_n n^m \\
	&= 2 \left( \omega_0 \sum_{n=1}^{12} n^{m}
	+ \omega_1 \sum_{n=1}^{12} n^{m + 1}
	+ \omega_2 \sum_{n=1}^{12} n^{m + 2}
	+ \omega_3 \sum_{n=1}^{12} n^{m + 3}
	+ \omega_4 \sum_{n=1}^{12} n^{m + 4}
	- \sum_{n=1}^{12} t_n n^m \right)
\end{align*}

よって、

\begin{align*}
	& \frac{\partial E}{\partial \omega_m} = 0 \\
	& \Leftrightarrow 
	\omega_0 \sum_{n=1}^{12} n^{m}
	+ \omega_1 \sum_{n=1}^{12} n^{m + 1}
	+ \omega_2 \sum_{n=1}^{12} n^{m + 2}
	+ \omega_3 \sum_{n=1}^{12} n^{m + 3}
	+ \omega_4 \sum_{n=1}^{12} n^{m + 4}
	- \sum_{n=1}^{12} t_n n^m  = 0 \\
	& \Leftrightarrow
	\omega_0 \sum_{n=1}^{12} n^{m}
	+ \omega_1 \sum_{n=1}^{12} n^{m + 1}
	+ \omega_2 \sum_{n=1}^{12} n^{m + 2}
	+ \omega_3 \sum_{n=1}^{12} n^{m + 3}
	+ \omega_4 \sum_{n=1}^{12} n^{m + 4}
	= \sum_{n=1}^{12} t_n n^m
\end{align*}

となる。
$m=0, \dots , 4$ で具体値を代入して、

\begin{align*}
	\begin{bmatrix}
    12 & 78 & 650 & 6084 & 60710 \\
	78 & 650 & 6084 & 60710 & 630708 \\
	650 & 6084 & 60710 & 630708 & 6735950 \\
	6084 & 60710 & 630708 & 6735950 & 73399404 \\
	60710 & 630708 & 6735950 & 73399404 & 812071910
	\end{bmatrix}
	\begin{bmatrix}
		\omega_0 \\
		\omega_1 \\
		\omega_2 \\
		\omega_3 \\
		\omega_4 
	\end{bmatrix}
	=
	\begin{bmatrix}
		182.9 \\
		1286.6 \\
		10362.6 \\
		90481.4 \\
		835744.2
	\end{bmatrix}
\end{align*}

これを解いて、

\begin{align*}
	\begin{bmatrix}
		\omega_0 \\
		\omega_1 \\
		\omega_2 \\
		\omega_3 \\
		\omega_4 
	\end{bmatrix}
	=
	\begin{bmatrix}
		11.120202 \\
		-9.293603 \\
		4.063967 \\
		-0.454027 \\
		0.014744
	\end{bmatrix}
\end{align*}

を得る。

(1)の式に$\omega$の値を代入して$E=12.144164724$を得る。

P61に「実際の誤差関数$ \bm{loss} $の最小値は約12」と書いてあるが、それと一致する。
\end{document}

